Graphics processing units (GPUs) are increasingly well designed for high-performance computing.
GPU programming environments have also become matured for general-purpose computing on GPUs (GPGPUs).
Originally, the GPU software stack is tailored to accelerate particular best-effort applications.
In recent years, however, GPU technologies have been applied for real-time systems, extending the operating system (OS) modules to support real-time GPU-resource management.
Unfortunately, such a system extension makes it difficult to maintain the system with version updates, since the OS kernel and device drivers need to be modified at the source-code level, preventing continuous research and development of GPU technologies for real-time systems.
A loadable-kernel-module (LKM) framework, called ``Linux-RTXG,'' for managing real-time GPU resources on Linux without having to modify the OS kernel and device drivers is proposed and experimentally evaluated.
Linux-RTXG provides novel mechanisms for interrupt interception and independent synchronization to achieve real-time scheduling and resource reservation capabilities for GPU applications on top of existing device drivers and runtime libraries.
The results of an experimental evaluation of Linux-RTXG demonstrate that the overhead incurred by introducing Linux-RTXG is comparable to that for introducing existing kernel-dependent approaches.
Moreover, they demonstrate that multiple GPU applications can be successfully scheduled by Linux-RTXG to meet their priority and quality-of-service (QoS) requirements in real-time.
