\section{Introduction}
% simple
% GPUによって加速されるリアルタイムシステム?
% GPUをリアルタイムに追加する?
GPUは汎目的アプリケーションを加速させるデバイスとして既に一般的になりつつある。
その応用範囲は自動運転に用いるナビゲーション\cite{cmu:routing}や障害物検知\cite{hirabayashi:cpsna2013}、
核融合炉に用いるTokamakの制御\cite{tokamak}、
ユーザインタラクティブなアプリケーション\cite{kato:rtas2011}、データベース\cite{bakkum:sql}まで多岐にわたっており、
ベンチマークsuit\cite{rodinia}に提供される。
これらの研究によってGPUの性能は既に実証されており、広いアプリケーションドメインに受け入れられている。

より普及した要因としてはGPUが大量のプロセッシングコアを用いてデータ並列性のあるアプリケーションを実行することで高速な処理が可能であること、
それらを支えるCUDA\cite{nvidia:cuda_zone}やOpenCL\cite{opencl}などの言語やランタイムなどが統合されたプラットフォームがベンダーによって提供され始めたことが背景にある

しかしながら公式ベンダーから提供されるランタイムシステムでは、
綿密な資源管理機能が保証されておらず、汎目的利用は可能かもしれないが、Multitasking systemでは利用が困難であることから、GPU資源管理ソフトウェアが必要となっている。
加えて、近年ではサイバーフィジカルシステムなどリアルタイム性を要求するシステムにおいてもGPUの利用が望まれており、汎用システム向けだけでなく、リアルタイムシステム向けの資源管理が求められる。

我々はこれまでに、いくつかのGPU resource management に関する研究を進めてきた。
TIMEGRAPH\cite{kato:timegraph}ではGPUに送信されるコマンドをeach atomic setにグルーピングしたGPU command groupを対象として、スケジューリングやリザベーションメカニズムを提供している。
RGEM\cite{kato:rgem}、Gdev\cite{kato:gdev}ではGPGPUにフォーカスし、GPUで実行される単位であるカーネルとデータ転送をスケジューリングしている。
これらの研究はリバースエンジニアリングによって提供されており、アーキテクチャの更新や、全ての機能の提供が困難であるといったデメリットが存在している。

GPUの機能の多くはAPIによって提供されており、ユーザアプリケーションからライブラリを通じて、デバイスドライバ、GPUへと処理が発行される。
そのためリアルタイムを前提としたシステムにおいて、真にその要件を満たすためには、GPUのリソースマネージメントだけでなく、
ホスト側のタスクについても管理してやる必要がある。
我々がホストとデバイス間のデータ転送時間について\cite{fujii:icpads2013}評価した研究では、CPU側で他のタスクがリソースを専有していた場合、
多量のレイテンシが発生することが実際に確認できている。

Elliottらの提案するGPUSync\cite{elliott:gpusync13}\cite{elliott:explor14}では資源管理システムを既存ランタイムから分離し、
アクセスの調停を行うことで、プロプライエタリ・ソフトウェアが行う資源管理をスルーした上で自身の資源管理を行っている。
GPUSyncでは上記のホスト側のタスクに関する資源管理についても言及しており、
GPU側の資源管理との組み合わせによる検証を行うためにコンフィギュアラブルなものを目指している。

しかしながらGPUSyncは$LITMUS^{RT}$\cite{litmus}上に実装されており、カーネルへの変更を多分に含んでいる。
Gdevについても同様にデバイスドライバ自体の変更を必要としている。
これらの変更の多くはパッチを利用して、ユーザへインストレーションを要求する形が用いられるが、
このパッチには開発者とユーザ両者に大きな負担を与える。
具体的には開発者は、常に最新のカーネルのリリースに追い付くために、パッチを維持していく義務がある。
しかしLinuxはその更新頻度が早く、開発者が最新のカーネルにむけてポーティング作業を完了させる前に、新しいバージョンのリリースが起きることが多い。そのためカーネルの選択について制限される傾向がある。

%cpu側のことについて
カーネルの修正を含む問題に対して我々はこれまでCPU側のタスクをスケジューリングするためのリアルタイム拡張としてRESCH\cite{kato:}を提供してきた.
RESCHはローダブルカーネルモジュールを利用しており、カーネルの内部関数を利用することで、カーネル自体にパッチを当てること無くリアルタイム拡張可能としている。
しかしながら、RESCHはGPUを保持するシステムのようなヘテロジニアスな環境には対応していない。

\textbf{Contribution:}
In this paper, we present linux real-time extension for cpu-gpu resouce coordination called linux-rtxg for cpu and gpu coordinated resource management, this extension is able to more easily re-configure the resource management policy and the installation.
Linux-RTXGの最大きな貢献は、OS機能を持ちながら、OSから独立することで、既存のGPU資源管理に関する研究において考慮されていないカーネル編集に伴う開発者、ユーザへの負担をなくしたことである。

加えて、既存研究において既に取り組まれている技術を含有し、
これまでのフレームワークからより簡易にコンフィギュア可能な構成を保持することで、
今後GPUが含まれるシステムにおける資源管理に関する研究を加速させることを最終目標とする。
本論文においては本手法利用時に発生するオーバヘッドを提示し、利用可能範囲内であることを証明する。

\TODO{細かい部分}

\textbf{Organization}
This paper rest of 
本論文は7章で構成される。
