\section{Introduction}
GPUは汎目的アプリケーションを加速させる手法として既に一般的になりつつある。
GPUはSIMD型であり、数千ものコアで大量のスレッドを並列処理することで、データ並列性を持ったアプリケーションにかかる時間を劇的に短縮させる。

GPUはその性能からCyber-Physical Systemで利用され始めている。
Cyber-Physial systems represent next generation networked and embedded systems.
These system was tightly coupled with computation and physical elements to control real-world phenomena.
There control algorithms are becoming more and more complex, which distinguishes CPS from traditional safety-critical embedded systems in terms of the computational cost.
From their factors, CPS requires more high-performance computing resources to achieve real-fast computing.

既にGPUはHBT-EP TokamakやAutomate driving carなどのCPSアプリケーションに利用され始めている。
HBT-EP Tokamakは核融合の制御に使われるシステムでセンサから取得したプラズマの状態をGPUで処理し、電磁波によって制御している。
Automate driving carに使える、障害物検知、Pedestrian Tracking, ルート推定などに使われている。

しかしながらGPUのようなメニーコアやマルチコアに関する技術は未だ成熟しておらず、
性能的な発展を見込めるが、ソフトウェア的な、特に資源管理という面から見ると未熟であり多くの問題を抱えている。
GPUは顕著にその傾向が見られる。
その理由はGPUのソフトウェアはベンダーによって提供されており、そのほとんどがクローズドソースなためである。
開発者はより綿密な資源管理を行うために、ソフトウェアの解析から取り組まなければならない。

我々もこれまでにGPUの資源管理としてGdevやRGEM、TimeGraphに取り組んできたが、
それらはリバースエンジニアリングによって支えられてきたシステムである。
加えて、それらのシステムが利用するデバイスドライバについても、Linuxを支えるNouveau Projectが管理するNouveauを利用しており、
こちらもリバースエンジニアリングによって成り立っている。

リバースエンジニアリングには、全ての機能の再現が難しいことや、バージョンアップへの追従が負担などの技術的な問題に加え、その方法次第では法律的な問題も抱えている。

Elliottらの提案するGPUSyncでは資源管理システムを既存ランタイムから分離し、
アクセスの調停を行うことで、既存ランタイムが行う資源管理をスルーした上で自身の資源管理を行っている。
しかしながらGPUSyncは$LITMUS^{RT}$上に実装されており、カーネルへの変更を多分に含んでいる。
Gdevについても同様にデバイスドライバ自体への変更を必要としている。
この変更はOSへパッチを当てることによるインストレーションを必要としている。
このパッチを当てる作業は開発者とユーザへ大きな負担を与える。
開発者は、常に最新のカーネルのリリースに追い付くために、パッチを維持していく義務がある。
しかしLinuxはその更新頻度が早く、開発者が最新のカーネルにむけてポーティング作業を完了させる前に、新しいバージョンのリリースが起きることが多い。
そのためカーネルの選択について制限される傾向がある。

本問題に対して我々はこれまでリアルタイム拡張suitとしてRESCHを提供してきた。
RESCHはローダブルカーネルモジュールを利用しており、カーネルの内部関数を利用することで、
カーネル自体にパッチを当てること無くリアルタイム拡張可能としている。
しかしながら、RESCHはGPUを保持するシステムのようなヘテロジニアスな環境には対応していない。

\textbf{Contribution:}
本論文では、GPUを保持するシステムにおいてユーザーレベル\footnote{OSにパッチを当てずにの意}で資源管理可能なことをExploreし、
その成果として、GPU資源管理システムとRESCHを統合したリアルタイムスケジューリング拡張フレームワークとしてLinux-RTXを提供する。

我々はまず、GPU実行を含んだタスクモデルについて解説し、GPUスケジューリングに必要な要件を示す。
その後要件を満たすための手法として、Interrupt Interceptを提案し、
そして達成したGPUスケジューラをCPUスケジューラであるRESCHに統合、
CPU-GPUスケジューリングが可能なフレームワークであるLinux-RTXを提供する。
本論文は最適なスケジューリング機構を提案するものでなく、
最適なスケジューリング機構を探索するためのシステムを提案するものである。


\textbf{Organization}
This paper rest of 
本論文は◯章で構成される。
次章では、対象とするシステムの説明。
3章では、先行研究を解説し、解決すべき点を列挙し、
4章においてそれらの点の解決について説明する。
5章では、Linux-RTXGを利用した際の各オーバヘッドについて計測し評価を行う。
6章で本Linux-RTXGの問題点と今後の展望について考察していく。
