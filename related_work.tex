\SECTION{Related Work}\label{sec:relatedwork}
\begin{table*}[t]
\begin{center}
\caption{Linux-RTXG vs Prior Work}
\label{tab:comp:prior}
%\begin{tabular}{|c|c|c|c|c|c|c|c|c|c|c|} \hline
\ifthesis
\scalebox{0.65}{
\fi
\begin{tabular}{ccccccccccc} \hline
 & CPU & CPU & GPU Priority & Budget & Data/Computing & Closed Source& Patchless & GPU Runtime \\ 
& Fixed-Priority & EDF & Scheduling & Enforcement & Overlap & Compatible &  & independent \\ \hline
 RGEM       &   &   & x &   &   & x & x &   \\ 
 Gdev       &   &   & x & x & x &   &   &  \\ 
 PTask      &   &   & x & x & x & x &  & x \\ 
 NEON & & & x &  & x & x &  & x \\
 GPUSync    & x & x & x & x & x & x &   & x \\ 
 GPU-Sparc   & x & & x &   & x & x & x &   \\ 
 %GPU-Sparc   & $B"%(B &   & x &   & x & x & x &   \\
 Linux-RTXG & x & x & x & x & x & x & x & x \\ \hline
\end{tabular}
\ifthesis
}
\fi
\end{center}
\vspace{-4mm}
\end{table*}
RGEM~\cite{kato:rgem} and GPU-Sparc~\cite{sparc} provide real-time GPU resource management without the OS kernel and device driver modifications.
However, their synchronization mechanism depends on proprietary software.
TimeGraph~\cite{kato:timegraph}, Gdev~\cite{kato:gdev}, Ptask~\cite{ptask}, and GPUSync~\cite{elliott:gpusync13} realize independent synchronization mechanisms but require modifying the OS kernel and device drivers.
To the best of our knowledge, Linux-RTXG is the only solution that can provide real-time GPU resource management with a synchronization mechanism, without modifying the OS kernel and device drivers.

Table~\ref{tab:comp:prior} shows a comparison of Linux-RTXG and previous work.
Note that Data/Computing Overlap indicates whether perform the data transfer and computing simultaneously, and Closed Source Compatible indicates whether use an NVIDIA  CUDA.
GPUSync supports the fixed-priority and the EDF scheduling policies for CPU tasks, while GPU-Sparc employs the $SCHED\_FIFO$ scheduling policy.
As shown from this table, Linux-RTXG has demonstrated all features shown in Table~\ref{tab:comp:prior}.
In particular, the resource management modules of Linux-RTXG are all loadable and are freed from the detailed implementation of runtime libraries, device drivers, and the OS kernel.

More in-depth resource management would require detailed information about the execution mechanisms in the black-box GPU stack. Menychtas et al. presented interception-based OS-level GPU scheduling, called NEON, without modifying software stack by using state machine inferencing~\cite{menychtas2013enabling},~\cite{neon}. 
GPU resource management using GPU microcontrollers~\cite{fujii:apsys2013} and in-kernel runtime functions~\cite{kato:gdev} has also been demonstrated to manage the GPU.
For these pieces of open-source work, the Nouveau project has been used as a baseline driver~\cite{nouveau}.
