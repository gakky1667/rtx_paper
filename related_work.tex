\SECTION{Related Work}\label{sec:relatedwork}
\begin{table*}[t]
\begin{center}
\caption{Linux-RTXG vs Prior Work}
\label{tab:comp:prior}
%\begin{tabular}{|c|c|c|c|c|c|c|c|c|c|c|} \hline
\ifthesis
\scalebox{0.65}{
\fi
\begin{tabular}{ccccccccccc} \hline
 & CPU & CPU & GPU Prio. & Budget & Data/Comp. & Closed Src.& Kernel& OS & GPU Runtime \\ 
& FP & EDF & Sched. & Enforcement & Ovlp. & Compatible & Free & independent & independent \\ \hline
 RGEM       &   &   & x &   &   & x & x & x &   \\ 
 Gdev       &   &   & x & x & x &   &   &   &   \\ 
 PTask      &   &   & x & x & x & x &   &   & x \\ 
 GPUSync    & x & x & x & x & x & x &   &   & x \\ 
 GPUSparc   & $B"%(B &   & x &   & x & x &   & x &   \\ 
 Linux-RTXG & x & x & x & x & x & x & x & x & x \\ \hline
\end{tabular}
\ifthesis
}
\fi
\end{center}
\end{table*}

RGEM~\cite{kato:rgem} and GPU-Sparc~\cite{sparc} provide real-time GPU
resource management without the OS kernel and device driver
modifications.
However, their synchronization mechanism depends on proprietary
software.
TimeGraph~\cite{kato:timegraph}, Gdev~\cite{kato:gdev},
Ptask~\cite{ptask}, and GPUSync~\cite{elliott:gpusync13} realize
independent synchronization mechanisms but require modifying the OS
kernel and device drivers.
To the best of our knowledge, Linux-RTXG is the only solution that can
provide real-time GPU resource management with a synchronization
mechanism, without modifying the OS kernel and device drivers.

Table~\ref{tab:comp:prior} shows a comparison of Linux-RTXG and previous
work.
GPUSync supports the fixed-priority and the EDF scheduling policies for
CPU tasks, while GPUSparc employs the $SCHED\_FIFO$ scheduling policy.
Note that Linux-RTXG has demonstrated all features shown in
Table~\ref{tab:comp:prior}.
In particular, the resource management modules of Linux-RTXG are all
loadable and are freed from the detailed implementation of runtime
libraries, device drivers, and the OS kernel.

More in-depth resource management would require detailed information
about the execution mechanisms in black-box GPU stacks.
Menychtas et al. presented enabling GPU resource management by inferring
interactions in the black-box GPU stack~\cite{menychtas2013enabling}. 
GPU resource management using GPU
microcontrollers~\cite{fujii:apsys2013} and in-kernel runtime
functions~\cite{kato:gdev} has also been demonstrated to manage the
GPU.
For these pieces of open-source work, the Nouveau project has been
used as a baseline driver~\cite{nouveau}.

%These works are very important, and we will also views of the further development of resource management their researches.

%In order to more growth the GPU as a powerful device,
%The challenges are efficiency compler, efficiency runtime, efficiency system such as treatment PCI.

%$B2f!9$O$3$l$^$G$$$/$D$+$N(BGPU$B$N;q8;4IM}$K4X$9$k8&5f(B~\cite{kato:timegraph,kato:rtas2011,kato:rgem,kato:gdev}$B$r9T$C$F$-$?!#(B
%TimeGraph$B$O(BGPU$B$KAw?.$5$l$k%3%^%s%I$r%9%1%8%e!<%j%s%0$9$k$3$H$G(BCUDA$B$K$+$.$i$:!"(BOpenGL$B$J$I!"(B
%$BA4$F$N(BGPU$B$rMxMQ$K4X$9$k;q8;4IM}$r9T$C$F$$$k!#(B
%$B$7$+$7$J$,$i(BGPU$B$N%3%^%s%I$O=hM}$N<B9T$@$1$G$J$/!"%G!<%?E>Aw!"3d9~$_=hM}EPO?$J$I$N=hM};~$K$bAw?.$5$l$F$*$j!"(B
%$BK\Ev$K%9%1%8%e!<%j%s%0$9$k$Y$-C10L$G$N%9%1%8%e!<%j%s%0$K$O8~$$$F$$$J$$$3$H$,$o$+$C$F$$$k!#(B
%$B$=$N$?$a!"(BRGEM$B$O(BGPGPU$B$KFC2=$7!"(BGPU$B%+!<%M%k<B9TC10L$G$N%9%1%8%e!<%j%s%0$rL\;X$7!"8GDjM%@hEY$G$N%9%1%8%e!<%j%s%0$r<B8=$7$F$k!#(B
%$B2C$($F!"%G!<%?E>Aw$N%;%0%a%s%HJ,$1$K$h$C$F%N%s%W%j%(%s%W%F%#%V$JFC@-$K$b$?$i$5$l$k%G%a%j%C%H$r:G>.8B$K$7!"(B
%$B%l%9%]%s%9%?%$%`$N8~>e$rL\;X$7$F$$$k!#(B
%Gdev$B$O(BRGEM$B$NH/E87A$G$"$j!"2>A[(BGPU$B$H(BResource Reservation$B$K$h$k(BQoS$B@)8f$d!"(BOS$B6u4V$G$N(BCUDA$B<B9T$J$I$r<B8=$7$F$$$k!#(B
%$B2C$($F%G!<%?E>Aw$H%+!<%M%k<B9T$r%*!<%P%i%C%W$5$;$k$3$H$G<B9T;~4V<+BN$N=L>.$r<B8=$7$F$$$k!#(B
%
%PTask~\cite{ptask} is an OS abstraction for GPU applications that optimizes data transfers and GPU scheduler.
%
%Elliott et al. present GPUSync~\cite{elliott2013gpusync,elliott:explor14}.
%GPUSync$B$G$O%[%9%H$+$i(BGPU$B$X$N%G!<%?E>Aw3+;O$+$i!"(BGPU$B$G$N=hM}!"(BGPU$B$+$i%[%9%H$X$N%G!<%?E>Aw$^$G$r%/%j%F%#%+%k%;%/%7%g%s$H@_Dj$7!"(B
%runtime$B$X$N%"%/%;%9$O%/%j%F%#%+%k%;%/%7%g%s$r6h@Z$j$H$7$FC10l$N%"%/%;%9$H$J$k$h$&$KD4Dd$r9T$C$F$$$k!#(B
%$B$3$l$K$h$C$F%/%m!<%:%I%=!<%9$J%i%s%?%$%`$rMxMQ$7$D$D!"<+?H$N(BGPU$B;q8;4IM}$r<B8=2DG=$H$7$F$$$k!#(B
%GPUSync$B$O%"%/%;%9D4Dd$N<jK!$H$7$F(Bk-exclusion lock$B$N3HD%$rMxMQ$7$F$$$k!#(B
%$B2C$($F3F(BGPU$B$4$H$K(BResource Reservation$B$K$h$k(BQoS$BC4J]$r9T$C$F$*$j!"3F(BGPU$B4V$N(BP2P migration$B$b<B8=$7$F$*$j!"(BMultipleGPU$B$X$N<+F03d$jEv$F$b9T$C$F$$$k!#(B

%Han et al. show GPU-SPARC~\cite{sparc}.
%GPU-Sparc support to automatically split and run the GPU kernel concurrently over multi-GPU, and then they supported priority queue based scheduling.

